\documentclass[12pt]{article}

% -------------------
% packages
% -------------------
\usepackage{setspace}
\usepackage[danish]{babel}
\usepackage[framemethod=TikZ]{mdframed}
\usepackage[utf8]{inputenc}
\usepackage{amsmath}
\usepackage{csquotes}% Recommended
\usepackage{varwidth}
\usepackage{verbatim}
\usepackage{url}
\usepackage{physics}
\usepackage{amsmath}
\usepackage{graphicx}
\usepackage[utf8]{inputenc}
\usepackage{siunitx}
\usepackage{amssymb}
\usepackage{accents}
\usepackage{tikz}
\usetikzlibrary{trees}
\usepackage[danish]{babel}
\usepackage[margin=3.5cm]{geometry}

\newcommand{\tss}[1]{\textsubscript{#1}}

% -------------------
% siunitx setup
% -------------------
\sisetup{exponent-product = \cdot}
\newcommand{\s}[1]{~\si{#1}}
\newcommand{\us}[1]{~\si{\qty[#1]}}
\newcommand{\kg}{\s{kg}}
\newcommand{\sek}{\s{s}}
\newcommand{\met}{\s{m}}
\newcommand{\metps}{\s{\frac{m}{s}}}
\newcommand{\newt}{\s{N}}
\newcommand{\hz}{\s{hz}}
\newcommand{\radia}{\s{rad}}

% -------------------
% centerverbatim
% -------------------
\newenvironment{centerverbatim}{%
  \par
  \centering
  \varwidth{\linewidth}%
  \verbatim
}{%
  \endverbatim
  \endvarwidth
  \par
}

\usepackage[explicit]{titlesec}

% -------------------
% biblatex
% -------------------
\usepackage[style=authoryear-ibid,backend=biber]{biblatex}
\bibliography{sample.bib}

% -------------------
% custom env
% -------------------
\usepackage{amsthm}
%\newtheorem{theorem}{Theorem}
\newtheorem{theorem}{Theorem}[subsection]
\newtheorem{corollary}{Korollar}[theorem]
\theoremstyle{definition}
\newtheorem{example}{Eksempel}[definition]
\newtheorem{lemma}[theorem]{Lemma}
\theoremstyle{remark}
\newtheorem*{remark}{Note}
\theoremstyle{definition}
\newtheorem{definition}{Definition}[section]
\renewcommand\qedsymbol{$\blacksquare$}
%

% -------------------
% texthøjde
% -------------------
\usepackage{setspace}
\renewcommand{\baselinestretch}{1.5} 

% -------------------
% box
% -------------------
\newcounter{theo}[section]\setcounter{theo}{0}
\renewcommand{\thetheo}{\arabic{section}.\arabic{theo}}
\newenvironment{theo}[2][]{%
  \refstepcounter{theo}%
  \ifstrempty{#1}%
  {\mdfsetup{%
      frametitle={%
        \tikz[baseline=(current bounding box.east),outer sep=0pt]
        \node[anchor=east,rectangle,fill=blue!20]
        {\strut #2};}}
  }%
  {\mdfsetup{%
      frametitle={%
        \tikz[baseline=(current bounding box.east),outer sep=0pt]
        \node[anchor=east,rectangle,fill=blue!20]
        {\strut Theorem~\thetheo:~#1};}}%
  }%
  \mdfsetup{innertopmargin=10pt,linecolor=blue!20,%
    linewidth=2pt,topline=true,%
    frametitleaboveskip=\dimexpr-\ht\strutbox\relax
  }
  \begin{mdframed}[]\relax%
    \label{#2}}{\end{mdframed}}

% -------------------
% parindent
% -------------------
\parindent{\noindent}
\usepackage{blindtext}
% -------------------
% titlepage
% -------------------
\title{
  {{Kvantemekanisk harmonisk svingning og hæve-/sænkeoperatorer}}\\
  {\small af}\\
  {\large{Louis Clément}}\\
  {\large Hillerød Tekniske Gymnasium}\\
  \vspace{2cm}
  {\includegraphics[scale=1.5]{gym.png}}
  \vspace*{1.5cm}
}
\author{
  \begin{tabular}[!h]{c}
    \textbf{SOP fag}\\
                Matematik A, Fysik A \\
    \textbf{SOP vejledere}\\
                 Mikkel Oglesby \\ Jacob Skytte Salgaard Bendtsen
  \end{tabular}
}
\date{\vspace*{1cm}\textit{6. december 2020}}

%\titleformat{\section}{\huge\textbf}{\titlebar{3em}{\thesection}}{0.1cm}{#1} \titleformat{\subsection}{\large\textbf}{\titlebar{5em}{\thesubsection}}{0.1cm}{#1}
\usepackage{atbegshi}% http://ctan.org/pkg/atbegshi
\AtBeginDocument{\AtBeginShipoutNext{\AtBeginShipoutDiscard}}
\numberwithin{equation}{section}

\begin{document}


\pagenumbering{gobble}
\maketitle
\newpage
\tableofcontents

\newpage
\setcounter{page}{1}
\pagenumbering{arabic}

\section{Gennemgang af matematiske metoder}
\subsection{Vektorrum}
\begin{definition}
  Et linært vektorrum $\mathbb V$ er defineret over et felt $\mathbb F$ (hvor vi stort set kun vil beskæftige os med $\mathbb C$). En \textit{vektor} $\ket X$ er da et element af $\mathbb V$. Disse elementer har følgende egenskaber
  \begin{itemize}
  \item En regel for $\ket V+\ket W$
  \item En regel for $a\ket V, \forall a\in \mathbb F$
  \item Distributiv skalarmultiplikation for vektorelementer\\$a(\ket V+\ket W) = a\ket V+a\ket W)$
  \item Distributiv skalarmultiplikation for skalarelementer\\$(a+b)\ket V = a\ket V + b\ket V$
  \item Associativ skalarmultiplikation\\$a(b\ket V) = ab\ket V$
  \item Abelsk addition\\$\ket V+\ket W=\ket W + \ket V$
  \item Associativ addition\\$\ket V +(\ket W + \ket Z) = (\ket V+\ket W) + \ket Z$
  \item Nulvektor, $\ket V + \ket 0 = \ket V$ samt et inverst element der kan fremstille en nulvektor ved addition $\ket V + \ket{-V} = \ket 0$
  \end{itemize}
\end{definition}

\begin{definition}
  Indre produktrum er et vektorrum $\mathbb V$ med et såkaldt indre produkt. Det indre produkt mellem to vektorer $V$ og $W$ betegnes $\braket{V}{W}$ og overholder
  \begin{itemize}
  \item Symmetri $\braket{V}{W} = \braket{W}{V}^*$ hvorledes $^*$ betegner komplekst konjugerede for $\mathbb C$
  \item $\braket{V}{V} \geq 0$ og $\braket{V}{V}=0\iff \ket V = \ket 0$
  \item $\braket{V}{(a\ket W + b\ket Z)} = \braket{V}{aW+bZ} = a\braket{V}{W}+b\braket{V}{Z}$
  \end{itemize}
  Det indre produkt kan defineres mere eksplicit som
  \begin{align}
    \label{eq:1}
    \begin{split}
    \braket{V}{W} &= \sum_i \sum_j v_i^* w_j \delta_{ij} \\
    &= \sum_i v^*_iw_i
    \end{split}
  \end{align}
  for en ortonormal basis, derfor $\delta_{ij}=\braket{i}{j}$
\end{definition}

\subsection{Linære operatorer}
\begin{definition}
En linær operator $\Omgea$ eller linær transformation $T$ er en funktion $T:\mathbb V_1 \to \mathbb V_2$ således at
\begin{align}
  \label{eq:2}
  T(cv_1+v_2) = c(Tv_1) + Tv_2
\end{align}

\begin{definition}
  En kommutator er defineret som
  \begin{align}
    \label{eq:com}
    [\Omega, \Lambda] = \Omega\Lambda - \Lambda \Omega
  \end{align}

\end{definition}

\begin{definition}
  Eigenpolynomiet findes som
  \begin{align}
    \label{eq:ei}
    P^n(\omega) = \sum_{m=0}^n c_w \omega^m = 0
  \end{align}
\end{definition}
Enhver operator i $\mathbb V^n(C)$ har $n$ eigenværdier. Eigenligningen er
\begin{align}
  \label{eq:kre}
  \det(\Omega-\omega \hat I) = 0
\end{align}
hvor $\hat I$ er identitetsoperatoren.

\begin{example}
  Rotationsoperatoren $R\qty(\frac{1}{2} \pi e_x)$ kan repræsenteres ved en matrix således
  \begin{align*}
    R\qty(\frac{1}{2} \pi e_x) \leftrightarrow \mqty[1&0&0\\0&0&-1\\0&1&0]
  \end{align*}
  Da kan vi skrive eigenligningen som
  \begin{align*}
    \det(\Omega-\omega \hat I) = \mqty|1-\omega &0&0\\0&-\omega&-1\\0&1&-\omega| = 0
  \end{align*}
\end{example}
Hvor eigenpolynomiet kan skrives som $(1-\omega)(\omega^2+1)=0$, og $\omega\in\qty{1,i,-i}$. 


\end{definition}
\newpage
\printbibliography

\end{document}
