\documentclass[12pt]{article}

% packages
% -------------------
\usepackage{setspace}
\usepackage[danish]{babel}
\usepackage[framemethod=TikZ]{mdframed}
\usepackage[utf8]{inputenc}
\usepackage{amsmath}
\usepackage{csquotes}% Recommended
\usepackage{varwidth}
\usepackage{verbatim}
\usepackage{url}
\usepackage{physics}
\usepackage{amsmath}
\usepackage{graphicx}
\usepackage[utf8]{inputenc}
\usepackage{siunitx}
\usepackage{amssymb}
\usepackage{accents}
\usepackage{tikz}
\usetikzlibrary{trees}
\usepackage[danish]{babel}
\usepackage[margin=3.5cm]{geometry}

\newcommand{\tss}[1]{\textsubscript{#1}}
\usepackage[scr=boondoxo]{mathalfa}
% -------------------
% siunitx setup
% -------------------
\sisetup{exponent-product = \cdot}
\newcommand{\s}[1]{~\si{#1}}
\newcommand{\us}[1]{~\si{\qty[#1]}}
\newcommand{\kg}{\s{kg}}
\newcommand{\sek}{\s{s}}
\newcommand{\met}{\s{m}}
\newcommand{\metps}{\s{\frac{m}{s}}}
\newcommand{\newt}{\s{N}}
\newcommand{\hz}{\s{hz}}
\newcommand{\radia}{\s{rad}}

% -------------------
% centerverbatim
% -------------------
\newenvironment{centerverbatim}{%
  \par
  \centering
  \varwidth{\linewidth}%
  \verbatim
}{%
  \endverbatim
  \endvarwidth
  \par
}

\usepackage[explicit]{titlesec}

% -------------------

% -------------------
\usepackage[style=authoryear-ibid,backend=biber]{biblatex}
\bibliography{sample.bib}

% -------------------
% custom env
% -------------------
\usepackage{amsthm}
%\newtheorem{theorem}{Theorem}
\newtheorem{theorem}{Theorem}[subsection]
\newtheorem{corollary}{Korollar}[theorem]
\theoremstyle{definition}
\newtheorem{example}{Eksempel}[definition]
\newtheorem{lemma}[theorem]{Lemma}
\theoremstyle{remark}
\newtheorem*{remark}{Note}
\theoremstyle{definition}
\newtheorem{definition}{Definition}[section]
\renewcommand\qedsymbol{$\blacksquare$}
%

% -------------------
% texthøjde
% -------------------
\usepackage{setspace}
\renewcommand{\baselinestretch}{1.25} 
\renewcommand{\H}{\mathcal H}
\renewcommand{\arraystretch}{0.75}

% -------------------
% box
% -------------------
\newcounter{theo}[section]\setcounter{theo}{0}
\renewcommand{\thetheo}{\arabic{section}.\arabic{theo}}
\newenvironment{theo}[2][]{%
  \refstepcounter{theo}%
  \ifstrempty{#1}%
  {\mdfsetup{%
      frametitle={%
        \tikz[baseline=(current bounding box.east),outer sep=0pt]
        \node[anchor=east,rectangle,fill=blue!20]
        {\strut #2};}}
  }%
  {\mdfsetup{%
      frametitle={%
        \tikz[baseline=(current bounding box.east),outer sep=0pt]
        \node[anchor=east,rectangle,fill=blue!20]
        {\strut Theorem~\thetheo:~#1};}}%
  }%
  \mdfsetup{innertopmargin=10pt,linecolor=blue!20,%
    linewidth=2pt,topline=true,%
    frametitleaboveskip=\dimexpr-\ht\strutbox\relax
  }
  \begin{mdframed}[]\relax%
    \label{#2}}{\end{mdframed}}

% -------------------
% parindent
% -------------------
\parindent{\noindent}
\usepackage{blindtext}
% -------------------
% titlepage
% -------------------
\title{
  {{Hæve-/sænkeoperatorer og hydrogenlampe}}\\
  {\large{Louis Clément}}\\
  {\large Hillerød Tekniske Skole}\\
  \vspace{2cm}
  {\includegraphics[scale=1.5]{gym.png}}
  \vspace*{1.5cm}
}
\author{
  \begin{tabular}[!h]{c}
    \textbf{SOP fag}\\
                Matematik A, Fysik A \\
    \textbf{SOP vejledere}\\
                 Mikkel Oglesby \\ Jacob Skytte Salgaard Bendtsen
  \end{tabular}
}
\date{\vspace*{1cm}\textit{6. december 2020}}

%\titleformat{\section}{\huge\textbf}{\titlebar{3em}{\thesection}}{0.1cm}{#1} \titleformat{\subsection}{\large\textbf}{\titlebar{5em}{\thesubsection}}{0.1cm}{#1}
\usepackage{atbegshi}% http://ctan.org/pkg/atbegshi
\AtBeginDocument{\AtBeginShipoutNext{\AtBeginShipoutDiscard}}
\numberwithin{equation}{section}

\begin{document}


\pagenumbering{gobble}
\maketitle
\newpage
\tableofcontents

\newpage
\setcounter{page}{1}
\pagenumbering{arabic}

\section{Gennemgang af matematiske metoder}
\subsection{Hilbertrum}

\begin{definition}
  Et Hilbertrum $\mathcal H$ er et komplet vektorrum over et felt $\mathbb F$ med associeret indre produkt. Vi beskæftiger os med rum, hvori det gælder at
  \begin{align*}
    \braket{\psi}{\psi} = \int \psi^*(x)\psi(x) ~ dx < \infty
  \end{align*}
  for $\ket \psi \in \mathcal H$. Vi kan konstruere et $\infty$-dimensionelt Hilbertrum ved at overveje et kontinuert basis med elementerne $\ket a$ navngivet med en kontinuer variabel $a$, normaliseret således at
  \begin{align}
    \label{eq:con}
    \braket{a}{\tilde a} = \delta(a-\tilde a)
  \end{align}
  hvilket betyder vi kan skrive
  \begin{align}
    \label{eq:sw}
    \ket \psi = \int \psi(a)\ket a~da
  \end{align}
  
\end{definition}


\begin{definition}
  En \textit{ket} vektor $\ket V$ betegner en vektor af et abstrakt
vektorrum. I et endeligdimensionelt vektorrum kan en ket vektor
repræsenteres som
  \begin{align}
    \label{eq:colv}
    \ket V \leftrightarrow \mqty[v_1\\v_2\\\vdots\\ v_n]
  \end{align}
\end{definition}

\begin{definition}
  En \textit{bra} vektor $\bra V$ betegner en et element af dual
vektorrum (dualrum). Den kan repræsenteres som det transponerede
konjugat af den ket vektor den er dual på
  \begin{align}
    \label{eq:ket}
    \ket V \leftrightarrow \mqty[v_1\\v_2\\\vdots\\v_n]
\leftrightarrow \mqty[v_1^*&v_2^*&\cdots&v_n^*]
\leftrightarrow \bra V
  \end{align}
  Dualrum $\mathcal H^*$ består af linære afbildninger $\H^*\to \H$,
defineret med det indre produkt for $\langle \phi,\cdot \rangle\in
\H^$ som $\langle \phi,\cdot\rangle : \psi \mapsto
\braket{\phi}{\psi}$.
\end{definition}


\subsection{Linære operatorer}
\begin{definition}
En linær operator $\Omgea$ eller linær transformation $T$ er en
funktion $T:\mathbb V_1 \to \mathbb V_2$ således at
\begin{align}
  \label{eq:2}
  T(cv_1+v_2) = c(Tv_1) + Tv_2
\end{align}
Igennem denne tekst vil de kun repræsenteres som matricer $M$, således
at $T(x)=Mx$. En linær operator kan i øvrigt repræsenteres som $\ket
\psi \bra \phi \in \mathcal H \otimes \mathcal H^*$.

\end{definition}

\begin{definition}
  En kommutator er defineret som
  \begin{align}
    \label{eq:com}
    [\Omega, \Lambda] = \Omega\Lambda - \Lambda \Omega
  \end{align}
  hvor $\Omega,\Lambda \in \H\otimes \H^*$. Hviss. $[\Omega, \Lambda]
= 0$ kommuterer operatorerne.

\end{definition}

\begin{definition}
Enhver operator i $\mathbb V^n(C)$ har $n$
eigenværdier. Eigenværdiligningen (en omskrevet version) er
\begin{align}
  \label{eq:eigen}
  (\Omega - \omega \hat I)\ket V = \ket 0
\end{align}
Betingelsen for eigenvektoren er
\begin{align}
  \label{eq:kre}
  \det(\Omega-\omega \hat I) = 0
\end{align}
hvor $\hat I$ er identitetsoperatoren. Vi kan omskrive
eigenværdiligningen (ved at projicere den på en basis $\bra i$) til
\begin{align}
  \label{eq:eig2}
  \begin{split}
   \bra i \Omega - \omega\hat I \ket V = 0 \\
   \sum_j (\Omega_{ij}-\omega \delta_{ij})v_j = 0 
  \end{split}
\end{align}

Sættes determinanten til 0 får vi karakterligningen 
\begin{align}
  \label{eq:karklig}
  \sum_{m=0}^n c_w \omega^m = 0
\end{align}
og karakterpolynomiet
\begin{align}
  \label{eq:ei}
  P^n(\omega) = \sum_{m=0}^n c_w \omega^m
\end{align}
\end{definition}

\section{Kvantemekanisk harmonisk svingning}
Tilstanden af et partikel er beskrevet med en tilstandsvektor $\ket
\psi\in \mathcal H$.
\subsection{Schrödinger-ligningen}
Vi kan skrive den tidsuafhængige Schrödinger-ligning (TISE) som
\begin{align}
  \label{eq:schr}
  \hat H \ket \psi = E \ket \psi 
\end{align}
Hvor $\hat H$ er Hamiltonoperatoren. Fra den klassiske formidling af energi, får vi at $\hat H$ kan repræsenteres som
\begin{align*}
  \hat H = \frac{\hbar ^2}{2m} \pdv[2]{x}+V(x)
\end{align*}
Hvilket giver vi kan skrive SE som
\begin{align*}
  \frac{\hbar^2}{2m} \pdv[2]{x}\psi(x)+V(x)\psi(x) = i\hbar \pdv{t} \psi(x)
\end{align*}


\newpage
\printbibliography

\end{document}
